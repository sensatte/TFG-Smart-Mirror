% !TEX root = ../proyect.tex
\chapter{Introducci\'on}\label{intro}
En este proyecto abordaremos todos los aspectos relacionados con el an\'alisis, dise{\~n}o y prototipado de un espejo inteligente, un espejo con funcionalidades programables a trav\'es de un microordenador.

Un espejo no es más que una capa de cristal que protege otra fina capa met\'alica, normalmente plata o aluminio. Es esta capa met\'alica lo que nos permite vernos reflejados cuando nos miramos en uno; pero en el caso de los espejos bidireccionales, esta capa es mucho m\'as fina. Tan fina de hecho, que si la estancia est\'a lo suficientemente iluminada, parte de la luz no es reflejada, y atraviesa el espejo hacia el otro lado.

Siguiendo este concepto, si en un espejo bidireccional a{\~n}adimos una pantalla en la parte posterior, esa parte se quedar\'ia mayormente oscurecida, lo cual nos permitir\'a mostrar diversas aplicaciones o 'widgets' como podrían ser la hora actual, el tiempo o la música que se está reproduciendo.

El objetivo, pues, es la construcción de un espejo inteligente utilizando un monitor, una lámina de espejo bidireccional colocada delante de éste, y una Raspberry Pi que ejecute el programa a desarrollar en Python, el cual estará organizado en m\'odulos que el usuario podr\'a configurar tanto en apariencia como en posici\'on, con el fin de que logre una disposici\'on personalizada y a su gusto.
